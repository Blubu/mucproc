% \iffalse meta-comment
%
% Copyright (C) 2017 by Marei Peischl <tex@mareipeischl.de>
%
% This work may be distributed and/or modified under the
% conditions of the LaTeX Project Public License, either version 1.3
% of this license or (at your option) any later version.
% The latest version of this license is in
%   http://www.latex-project.org/lppl.txt
% and version 1.3 or later is part of all distributions of LaTeX
% version 2005/12/01 or later.
%
% This work has the LPPL maintenance status `maintained'.
%
% The Current Maintainer of this work is Marei Peischl.
%
% This work consists of the files mucproc.dtx and mucproc.ins
% and the derived files mucproc.cls and mucfontsize10pt.clo.
%
% \fi
%
% \iffalse
%
%<*!font-clo>
\def\mucprocVersion{1.01}
%<*driver>
\ProvidesFile{mucproc.dtx}[2017/02/22 v\mucprocVersion\ MuC-conference proceedings]
%</driver>

%<*class>
\NeedsTeXFormat{LaTeX2e}[2014/05/01]
\ProvidesClass{mucproc}
    [2017/02/20 v\mucprocVersion\ conference proceedings for the German MuC-conference]
%</class>
%
%<*driver>
\RequirePackage{scrlfile}
\ReplaceClass{article}{scrartcl}
\documentclass[english]{ltxdoc}
\usepackage{selinput}\SelectInputMappings{adieresis={ä},germandbls={ß}}
\usepackage[T1]{fontenc}
\usepackage{babel}
\EnableCrossrefs
\CodelineIndex
\RecordChanges
\usepackage{xparse}
\begin{document}
  \DocInput{mucproc.dtx}
\end{document}
%</driver>
% \fi
%
%
% \CharacterTable
%  {Upper-case    \A\B\C\D\E\F\G\H\I\J\K\L\M\N\O\P\Q\R\S\T\U\V\W\X\Y\Z
%   Lower-case    \a\b\c\d\e\f\g\h\i\j\k\l\m\n\o\p\q\r\s\t\u\v\w\x\y\z
%   Digits        \0\1\2\3\4\5\6\7\8\9
%   Exclamation   \!     Double quote  \"     Hash (number) \#
%   Dollar        \$     Percent       \%     Ampersand     \&
%   Acute accent  \'     Left paren    \(     Right paren   \)
%   Asterisk      \*     Plus          \+     Comma         \,
%   Minus         \-     Point         \.     Solidus       \/
%   Colon         \:     Semicolon     \;     Less than     \<
%   Equals        \=     Greater than  \>     Question mark \?
%   Commercial at \@     Left bracket  \[     Backslash     \\
%   Right bracket \]     Circumflex    \^     Underscore    \_
%   Grave accent  \`     Left brace    \{     Vertical bar  \|
%   Right brace   \}     Tilde         \~}
%
%
% \changes{v1.0}{2017/02/20}{Initial version}
%
% \GetFileInfo{mucproc.dtx}
%
% \DoNotIndex{\newcommand,\newenvironment}
%
%
% \title{The \textsf{mucproc} class\thanks{This document
%   corresponds to \textsf{mucproc}~\fileversion, dated \filedate.}}
% \author{Marei Peischl \\ \texttt{tex@mareipeischl.de}}
% \date{February 14, 2017}
% \maketitle
%
% \section{Introduction}
%
% The \textsf{mucproc} class provides a document class which allows to use the \LaTeX{} document preparation system for contributions to the German ``Mensch und Compuer 2017'' conference.
% It corresponds to the Word-template offered via the conference's homepage \url{http://muc2017.mensch-und-computer.de/} for the year 2017.
% By offering this document class and an additional usage demonstration file (\texttt{mucproc\_demo.tex}) we want to offer an easy to use possibility to prepare conference contributions in \LaTeX.
%
% The goal of this document class is to create PDF-files which look like those created with the Word-template.
% Most of the contributions are still expected to be typeset in Word, that's why the \textsf{mucproc} class uses the historically grown formatting guidelines which have been created for Word-Users.
% Therefore the settings of this document class might not always be the typographically best.
%
% \section{Usage}
% A complete example of use can be found on the github repository for this project: \url{https://github.com/Blubu/mucproc/}
%
% \textsf{mucproc} already configures all features needed for the usually required elements of a contribution to the MuC-conference. It loads a bunch of packages to simplify this process. Though the Syntax of the document using this class doesn't differ from classic \LaTeX{} documents.
%
% \subsection{Class options}
% \DescribeMacro{\documentclass}
% Start your document by loading the \textsf{mucproc} class with an option for the document's main language.
% The class will expect either \texttt{ngerman} or \texttt{english} as main language.
% Otherwise an error message will be thrown.
%
% For the language settings this class uses the \textsf{babel} package and loads \textsf{csquotes} for context sensitive quotations. The language option will also be used to configure the bibliography.
%
% The other class option which can be used by authors is \texttt{anonymous=\textit{truth value}}. It activates the anonymous-mode which shall be used for the submission to the review process before a contribution may be accepted.
% This option disables the author-field for the title and the additional author information as will be described later.
%
% The third type of class option is not meant to be used by the authors themselves.
% It rather provides a possibility to force a document to be compiled in the final mode and enables some additional features of the document class.
% The option \texttt{forcefinal=\textit{truth value}} will be used for a recompilation process for the preparation of the final version for the publication process of the proceedings.
% It can be passed to the class via command line interface and will disable a manually set anonymous-mode.
%
% \subsection{Title creation}
%
% \DescribeMacro{\title}
% The |\title| command behaves identically to most popular document classes. There may be line breaks created with |\\| or |\newline| inside the argument.
% Other formatting macros shall not be used, since this argument is also used for the metadata of the final PDF-file.
%
% \DescribeMacro{\author}
% \DescribeMacro{\and}
% The |author| macro can also be used similar to standard \LaTeX{} documents.
% If the document has more than one author, the names shall be separated with |\and|.
% \DescribeMacro{\thanks}
% The |\thanks| command shall be used to add the corresponding institution.
% It can be used similar to the standard \LaTeX{} definition, though it also can process an optional argument.
% The optional argument takes a label to reference an institution of another author.
% This prevents multiple mention of institutions.
% \DescribeMacro{\thanksref}
% This macro is used to reference the labels set with the |\thanks| command.
% The implementation of the institution reference is very different from the usual label/ref mechanism.
% That's why the reference is only defined after the label was set. If the argument of |\thanksref| isn't a known Label an error will occur.
%
% \begin{verbatim}
% \author{John Doe\thanks[inst:1]{Institution 1}\and
%    Jane Doe\thanks{Institution 2}\and
%    John Doe II\thanksref{inst:1}
% \end{verbatim}
% In contrast to the standard classes, \textsf{mucproc} doesn't place the footnotes created by |\thanks| on the bottom of the page, they are positioned directly below the author field of the title.
%
% Other possible entries for the title (e.\,g. |\date|)are ignored.
%
% \DescribeMacro{\maketitle} The title has to be created by calling the |\maketitle| command as usual.
%
% \subsection{Additional author information}
% For the final version the authors of contributions have the possibility to add additional information about all authors. For this the \textsf{mucproc} class provides the |authoraddendum| environment.
%
% \DescribeEnv{authoraddendum}
% This environment can take up to 2 arguments, one optional containing the path to a corresponding photograph and a mandatory one for the Authors name.
% \begin{verbatim}
% \begin{authoraddendum}[john_doe.png]{Doe, John}
% Text describing the author.
% \end{authoraddendum}
% \end{verbatim}
% If the optional picture argument ist left empty the text will be typeset over the whole |\textwidth|, otherwise there will be a 3\,cm column on the left side containing the photograph.
%
% \StopEventually{\PrintChanges\PrintIndex}
%
% \section{Implementation}
% This section contains the source code to the \textsf{mucproc.cls}. It will be tried to explain how the class breaks the tasks.
% \subsubsection{Compatibility mode}
% This document class intends to be compiled on as many different distributions as possible. Basically this class compiles on releases newer than 2014/05/01 without errors.
% Nevertheless some specific fallback tests require a the \textsf{expl3} package version of 2015/09/07 or newer.
% A compatibility mode test is implemented first and deactivates all problematic features of the class.
%    \begin{macrocode}
\RequirePackage{expl3}
\ExplSyntaxOn
\bool_new:N \g_MUC@compatibility_mode
\@ifpackagelater{expl3}{2015/09/07}{
	\bool_gset_false:N \g_MUC@compatibility_mode
}{
	\msg_new:nnn {mucproc}{compatibility}{
    Your~ LaTeX-Distribution~ seems~ to~ be~ outdated.\\\\
    Enabling~ compatibility~ Mode.\\\\
    Some~ features~ won't~ be~ available.
    }
    \msg_warning:nn {mucproc}{compatibility}
	\bool_gset_true:N \g_MUC@compatibility_mode
}

\RequirePackage{xparse}
%    \end{macrocode}
% The boolean |\g_MUC@compatibility_mode| is set true, if the \textsf{expl3} is older than required.
% \subsection{Class options}
% The key-value class options are implemented using the \textsf{scrbase} package defining a Family called \texttt{MUC}.
%    \begin{macrocode}
\RequirePackage{scrbase}
\DefineFamily{MUC}
\DefineFamilyMember{MUC}
\bool_new:N \g_MUC@pdfa_bool
%    \end{macrocode}
% The Options |anonymous| and |forcefinal| are implemented using the basic \textsf{scrbase} mechanism for boolean options:
%    \begin{macrocode}
\FamilyBoolKey{MUC}{anonymous}{MUC@anonymous}
\FamilyBoolKey{MUC}{forcefinal}{MUC@final}
\FamilyBoolKey{MUC}{headline}{MUC@headline}
\FamilyBoolKey{MUC}{copyrightinfo}{MUC@copyrightinfo}
\FamilyBoolKey{MUC}{pagestyle}{MUC@pagestyle}
%    \end{macrocode}
% Set up the default boolean values before processing options.
%    \begin{macrocode}
\MUC@headlinetrue
\MUC@copyrightinfotrue
\MUC@pagestyletrue

\FamilyProcessOptions{MUC}
%    \end{macrocode}
% The |pagestyle| settings can be disabled if |headline| and |copyrightinfo| are set false.
%    \begin{macrocode}
\ifMUC@headline
\else
    \ifMUC@copyrightinfo
    \else
        \MUC@pagestylefalse
    \fi
\fi
%    \end{macrocode}
% If the |forcefinal| option is set, all global options containing the substring ``draft'' or both of the substrings``final'' and ``false'' are removed to avoid any global draft mode.
%    \begin{macrocode}
\ifMUC@final
    \bool_if:NTF \g_MUC@compatibility_mode {
        \msg_new:nnn {mucproc} {compatibility-final-error} {
        forcefinal~ is~ not~ possible~ in~ compatibility~ mode.\\\\
        Update~ your~ TeX~ distribution~ and retry!}
        \msg_error:nn {mucproc} {compatibility-final-error}

    }{
    \sys_if_engine_luatex:F {
        \msg_new:nnn {mucproc} {luatex-required} {
        forcefinal~ requires~ LuaLaTeX~ for~ correct~ PDF/A~ output.\\\\
        Remove~ forcefinal~ option~ or~ change~ the~ compiler.
        }
        \msg_error:nn {mucproc} {luatex-required}
    }
    \clist_new:N \l_MUC@globalopts_clist

    \msg_new:nnn {mucproc} {force-final} {
        forcefinal~ option~ is~ enabled.\\\\
        Ignoring~ all~ global~ complementary~ options.}
    \msg_warning:nn {mucproc} {force-final}
    \msg_new:nnn {mucproc} {remove-draft}{
        forcefinal~ enabled:~ The~ global~ Option~ #1~ is~ ignored,~
        because~ it~ seems~ to~ activate draft~ mode!}

    \exp_args:Nx \clist_map_inline:nn {\@classoptionslist} {
    	\tl_if_in:nnTF {#1} {draft}
            {\msg_warning:nnx {mucproc} {remove-draft} {#1}}
            {\clist_put_right:Nn  \l_MUC@globalopts_clist {#1}}
        \tl_if_in:nnTF {#1} {final}
            {\tl_if_in:nnTF {#1} {false}
                {\msg_warning:nnx {mucproc} {remove-draft} {#1}}
                {\clist_put_right:Nn  \l_MUC@globalopts_clist {#1}}
            }
            {\clist_put_right:Nn  \l_MUC@globalopts_clist {#1}}
    }


    \edef\@classoptionslist{\clist_use:Nn \l_MUC@globalopts_clist {,}}
%    \end{macrocode}
% Additionally enable PDF/A-mode and metadata generation and disable anonymization.
%    \begin{macrocode}
    \bool_gset_true:N \g_MUC@pdfa_bool
    \MUC@anonymousfalse
    }
\fi
%    \end{macrocode}
% \subsection{Basic Setup}




% The next step is loading \textsf{scrartcl} as base class. According to the formatting rules we pass some options and use the \texttt{mucfontsize10pt.clo} file for setting up the font sizes.
%    \begin{macrocode}
\newcommand*{\@fontsizefilebase}{mucfontsize}
\PassOptionsToClass{abstract=true,parskip=full,fontsize=10pt,
    twoside=semi}{scrartcl}
\LoadClass{scrartcl}
\setparsizes{0pt}{7pt plus 7 pt}{1em plus 1 fill}
\raggedbottom
%    \end{macrocode}
%
% Now the other required packages can be loaded.
%    \begin{macrocode}
\RequirePackage{graphicx}
\RequirePackage{babel}
\PassOptionsToPackage{babel}{csquotes}
\bbl@frenchspacing
\RequirePackage{csquotes}

\RequirePackage{geometry}% setup margins
\geometry{paper=a4paper,top=4.8cm, bottom=5.7cm,left=3.8cm,right=4cm,
    head=12.045pt,headsep=\dimexpr1cm-12.045pt\relax}
%    \end{macrocode}

% \subsection{Bibliography setup}
% The bibliography will be created using \textsf{biblatex} with \textsf{biber} as backend.
%    \begin{macrocode}
\PassOptionsToPackage{backend=biber}{biblatex}
\PassOptionsToPackage{citestyle=authoryear-comp}{biblatex}
\PassOptionsToPackage{bibstyle=apa}{biblatex}
\RequirePackage{biblatex}
%    \end{macrocode}
% The APA-style requires a Language mapping. It is set for both supported main languages.
%    \begin{macrocode}
\DeclareLanguageMapping{english}{english-apa}
\DeclareLanguageMapping{ngerman}{ngerman-apa}
%    \end{macrocode}
% Set up additional Settings to get an equal output, like the Word-version.
%    \begin{macrocode}
\defcaptionname{ngerman}{\refname}{Literaturverzeichnis}

\renewcommand*{\nameyeardelim}{\addcomma\space}
%    \end{macrocode}
%
% \subsection{Font Choice}
% The formatting guidelines request Times New Roman as font. Since on most operating systems it is installed as a system font.
% We check if it is present and the currently used compiler supports the \textsf{fontspec} package.
%    \begin{macrocode}
\bool_new:N \g_MUC@fontspec_bool
%    \end{macrocode}
% These tests basically use the \textsf{expl3} methods, but also a fallback for older distributions is implemented to support older distributions. First we implement the compatibility-version.
%    \begin{macrocode}
\bool_if:NTF \g_MUC@compatibility_mode {
%    \end{macrocode}
% To test if PDF\TeX{} is used, we check if |\pdfmatch| is defined.
%    \begin{macrocode}
	\expandafter\ifx\csname pdfmatch\endcsname\relax
    \else
      \PassOptionsToPackage{utf8}{inputenc}
      \RequirePackage{inputenc}
	\fi
    \bool_gset_false:N \g_MUC@fontspec_bool
%    \end{macrocode}
% Next step is the mechanism to check if the used compiler supports \textsf{fontspec}.
%    \begin{macrocode}
}{
	\sys_if_engine_luatex:TF {
        \bool_gset_true:N \g_MUC@fontspec_bool
    }{
	    \sys_if_engine_xetex:TF{
            \bool_gset_true:N \g_MUC@fontspec_bool
	    }{
		    \sys_if_engine_pdftex:T {
            \bool_gset_false:N \g_MUC@fontspec_bool
		    \PassOptionsToPackage{utf8}{inputenc}
		    \RequirePackage{inputenc}
	        }
        }
    }
}
%    \end{macrocode}
% The font fallback is saved in a macro. This prevents the code to be written twice. It is executed when either \textsc{fontspec} is not supported or Times New Roman isn't installed.
%    \begin{macrocode}
\cs_new:Nn \MUC@font_Fallback: {
	\PassOptionsToPackage{T1}{fontenc}
	\RequirePackage{fontenc}
    \IfFileExists{txfonts.sty}{\RequirePackage{txfonts}}{}
}

\if_bool:N \g_MUC@fontspec_bool
	\RequirePackage{fontspec}
\fontspec_font_if_exist:nTF {Times~ New~ Roman}
    {\setmainfont{Times~ New~ Roman}}
	{\MUC@font_Fallback:}
\else:
	\MUC@font_Fallback:
\fi:
%    \end{macrocode}

% \subsection{PDF/A-mode and metadata creation}
% The PDF archiving mode is created using the \textsf{pdfx} package. For the correct creation the PDF/A format and the \texttt{.xmpdata} file for metadata specific system settings may be required.
% Therefore this mode is only active if the |forcefinal| mode is enabled.
%    \begin{macrocode}
\if_bool:N \g_MUC@pdfa_bool
    \PassOptionsToPackage{a-1b}{pdfx}
    \RequirePackage{pdfx}
    \hypersetup{hidelinks}
    \iow_new:N \MUC@xmpdata_stream
    \iow_open:Nn \MUC@xmpdata_stream {\jobname.xmpdata}
    \tl_new:N \g_MUC@xmp_title
    \tl_new:N \g_MUC@xmp_author

    \newcommand*{\MUC@passTitleData}{
    \begingroup
    \def\newline{}
    \def\\{}
    \xdef\MUC@author{\@author}
    \xdef\MUC@title{\@title}
    \cs_set:Npn \and {\exp_not:n {\exp_not:N \sep}}
    \tl_gset:Nx \g_MUC@xmp_title {\@title}
    \tl_gset:Nx \g_MUC@xmp_author {\@author}
    \endgroup
    \iow_now:Nx \MUC@xmpdata_stream
    	{
    		\exp_not:N \Title{\tl_use:N \g_MUC@xmp_title}
    		^^J
    		\exp_not:N \Author{\tl_use:N \g_MUC@xmp_author}
    		^^J
    		\exp_not:N \Org{Mensch~ und~ Computer~ 2017}
    	}
    }
%    \end{macrocode}
% If no PDF/A shall be created, the |\maketitle| data will only be passed to the headline.
%    \begin{macrocode}
\else:
    \msg_new:nnn {mucproc} {no-pdfa}{The~ mucproc~ class~ won't~ create~ PDF/A-mode.}
    \msg_info:nn {mucproc} {no-pdfa}
    \newcommand*{\MUC@passTitleData}{
        \begingroup
    	\def\newline{}
    	\def\\{}
    	\xdef\MUC@author{\@author}
    	\xdef\MUC@title{\@title}
        \endgroup
    }
\fi:
%    \end{macrocode}
%
% \section{Title creation}
% The title creation process also invokes the mechanism to pass the title data to headline and metadata.
% It can be initialized calling |\maketitle|.
% \subsection{Institution footnotes}
% The MuC formatting guidelines request the institutions typeset similar to footnotes, though they shouldn't be placed at the bottom of the page.
% The standard footnote mechanism is quite complicated, though we use the \LaTeX3 clist datatype to create a special mechanism implementing these ``footnotes''.
% The mechanism is called inside |\maketitle| but created outside of that macro definition to leave the code more readable.
%
% All Arguments of the |\thanks| command will be saved in the comma list called |\l_MUC@thanks_clist|.
%    \begin{macrocode}
\clist_new:N \l_MUC@thanks_clist
%    \end{macrocode}
% The value of the footnote counter has to be expanded when the |\thanks| macro is called. Otherwise the footnotemarks might not match.
% To achieve this expansion order the |\MUC@thanks:nn| command takes the counter value as first argument and the institution name as second.
%    \begin{macrocode}
\cs_set:Nn \MUC@thanks:nn {
    \clist_gput_right:Nn \l_MUC@thanks_clist
        {{#2\def\@thefnmark{#1}\@makefnmark}
    }}
%    \end{macrocode}
% Inside the real |\thanks| macro the first argument will be expanded directly.
% Additionally |\MUC@thanks| sets its own first (optional) argument as a label for the property list |\g_MUC@InstRef_prof|.
% This allows the footnote counter value to be accessed through the label.
%    \begin{macrocode}
\bool_if:NTF \g_MUC@compatibility_mode
{\NewDocumentCommand{\MUC@thanks}{om}{
    \footnotemark%
    \exp_args:Nx \MUC@thanks:nn {\thefootnote}{#2}
    }
}{
\NewDocumentCommand{\MUC@thanks}{om}{
    \footnotemark%
    \IfNoValueF{#1}{
        \prop_if_exist:NF \g_MUC@InstRef_prop {
            \prop_new:N \g_MUC@InstRef_prop}
        \prop_put:Nnx \g_MUC@InstRef_prop {#1} {\thefootnote}
        }
    \exp_args:Nx \MUC@thanks:nn {\thefootnote}{#2}
    }
}
%    \end{macrocode}
% The mechanism seems to work like the typical \LaTeX{} label/ref though no backref is supported.
% To inform the user about this an error will be thrown when a label is accessed which does not exist.
%    \begin{macrocode}
\bool_if:NTF \g_MUC@compatibility_mode {
\msg_new:nnn {mucproc} {thanksref-unavailable} {
    Compatibilty~ mode~ enabled.\\\\
    The~ institution~ reference~ mechanism~ is~ not available.\\\\
    Update~ Your~ TeX~ distribution~ to~ be~ able~
        to~ use~ this~ feature.
    }
\msg_warning:nn {mucproc} {thanksref-unavailable}
    \newcommand\MUC@thanksref[1]{$\dagger$}
}{
\msg_new:nnn {mucproc} {thanks-label-missing} {
    No~ institute~ labeled~ '#1'.\\\\
    Add~ a~ label~ using~ the~ optional~ argument~
        of~ the~ \thanks~ command.
    }

\NewDocumentCommand\MUC@thanksref{m}{
    \prop_if_in:NnTF \g_MUC@InstRef_prop {#1} {
    \def\@thefnmark{\prop_item:Nn \g_MUC@InstRef_prop {#1}}%
    \@makefnmark
    }
    {\msg_error:nn {mucproc} {thanks-label-missing}}
}
}
%    \end{macrocode}
% After the title has been set, the institutional notes shall be typeset.
%    \begin{macrocode}
\newcommand*\typeset@MUC@thanks {
    \clist_use:Nn \l_MUC@thanks_clist {\rule{0pt}{0pt}\\}}
%    \end{macrocode}
% \subsection{Typesetting the title}
% As written before, |\maketitle| has to execute additional jobs.
% Since the title fields also can contain other macros or even line breaks, the expansion process has to be controlled carefully.
%    \begin{macrocode}
\renewcommand*{\maketitle}{
\begingroup
%    \end{macrocode}
% Before passing the data to the headlines or metadata using the earlier defined |\MUC@passTitleData| we need some redefinitions for the commands, which can be used inside the title fields.
%    \begin{macrocode}
\DeclareExpandableDocumentCommand{\thanks}{om}{}
\def\thanksref##1{}
\def\and{\unskip,~\ignorespaces}
\MUC@passTitleData
%    \end{macrocode}
% Not the actual title creation begins redefining the user macros to those prepared for that purpose.
%    \begin{macrocode}
\let\titlepage@restore\relax
\renewcommand*\thefootnote{\@arabic\c@footnote}
\let\thanks\MUC@thanks
\let\thanksref\MUC@thanksref
%    \end{macrocode}
% Now only the title fields have to be typeset.
%    \begin{macrocode}
\global\@topnum=\z@
\setparsizes{\z@}{\z@}{\z@\@plus 1fil}\par@updaterelative
\null
\vskip 41pt
	{\usekomafont{title}{\Huge \@title \par}}%
    {\ifx\@subtitle\@empty\else\vskip .5em\usekomafont{subtitle}%
    \Large\@subtitle\par\fi}%
    \vskip 7pt
	{%
		\usekomafont{author}{%
			\lineskip .5em%
				\@author
		\par
		}%
	}%
\par
\vskip 8pt
\bool_if:NTF \g_MUC@compatibility_mode
    {\thispagestyle{empty}}
    {\thispagestyle{MUC@titlepagestyle}}
\typeset@MUC@thanks
\par
\vspace{8pt}
\endgroup
\setcounter{footnote}{0}%
\global\let\and\relax
\global\let\thanksref\relax
}
%    \end{macrocode}

% \subsection{Anonymization and additional author information}
% The anonymization is realized checking the setting using the corresponding if-expression.
% The |\@author| information to the language dependent meaning of |\MUCanonymousauthor|.
%    \begin{macrocode}
\newcaptionname{ngerman}{\MUCanonymousauthor}{Anonymisierte Fassung}
\newcaptionname{english}{\MUCanonymousauthor}{anonymous version}
\ifMUC@anonymous
    \def\@author{\MUCanonymousauthor}
    \def\author#1\relax
%    \end{macrocode}
% The |authoraddendum| environment shall create no output, if the anonymous mode is active.
%    \begin{macrocode}
    \RequirePackage{comment}
    \excludecomment{authoraddendum}
\else
%    \end{macrocode}
% \begin{environment}{authoraddendum}
% The Implementation works via the typical xparse mechanisms.
%    \begin{macrocode}
    \NewDocumentEnvironment{authoraddendum}{d<>om}{
        \par
        \IfNoValueTF{#2}{
        {\usekomafont{authoraddendum}#3}\par\medskip
        }{
        \begin{minipage}[t]{3cm}
        \vspace{0pt}
        \IfNoValueTF{#1}{
            \includegraphics[width=\linewidth]{#2}
        }{
            \includegraphics[#1]{#2}
        }
        \end{minipage}\hspace{.5cm}%
        \begin{minipage}[t]{\dimexpr\linewidth-3.5cm\relax}
        \vspace{0pt}
        {\usekomafont{authoraddendum}#3}\par\medskip%
        }
        \usekomafont{authoraddendum}
        }{
        \IfNoValueF{#2}{\end{minipage}}
        \par
        }
\fi
%    \end{macrocode}
% \end{environment}
%
%

% \subsection{Visible adjustment of sizes and spacing}
% To match the Word template we have to adjust a bunch of spacing settings as well as font sizes. The easiest way to achieve this is using the KOMA-Script mechanisms.
% \subsection{Structure Settings}
% Word indents the section headings independent from the width of the section number. To achieve this a macro called |\MUC@sectionnumberbox| is created and all section formats are redefined.
% In compatibility mode these settings will be ignored.
%    \begin{macrocode}
\reverse_if:N  \if_bool:N \g_MUC@compatibility_mode
\newcommand*{\MUC@sectionnumberbox}[1]{\makebox[28bp][l]{#1\autodot}}
\renewcommand*{\sectionformat}{
    \MUC@sectionnumberbox{\thesection}}
\renewcommand*{\subsectionformat}{
    \MUC@sectionnumberbox{\thesubsection}}
\renewcommand*{\subsubsectionformat}{
    \MUC@sectionnumberbox{\thesubsubsection}}
%    \end{macrocode}
% The spacing around the headings also has to be adjusted:
%    \begin{macrocode}
\RedeclareSectionCommand[afterskip=1.5ex plus .2ex,%
    beforeskip=-3.6ex plus -1ex minus -.2ex]{section}
\RedeclareSectionCommand[afterskip=.5ex plus .2ex,%
    beforeskip=-1ex plus -1ex minus -.2ex]{subsection}
\RedeclareSectionCommand[afterskip=.5ex plus .2ex,%
    beforeskip=-1ex plus -1ex minus -.2ex]{subsubsection}
\RedeclareSectionCommand[afterskip=1em,%
    beforeskip=1ex plus 1ex minus .2ex]{paragraph}
\RedeclareSectionCommand[afterskip=1em,%
    beforeskip=1ex plus 1ex minus .2ex]{subparagraph}
\fi:
\ExplSyntaxOff
%    \end{macrocode}
% The font settings will be done together with all other font settings in the next subsection.
%
% \subsection{Font Settings}
% Adjusting the font settings to match the Word version:
%    \begin{macrocode}
\setkomafont{disposition}{}
\setkomafont{title}{}
\setkomafont{author}{\large}
\setkomafont{section}{\LARGE}
\setkomafont{subsection}{\Large}
\setkomafont{caption}{\footnotesize\itshape}
\setkomafont{captionlabel}{\usekomafont{caption}}
\newkomafont{abstractcontent}{\small}
\newkomafont{abstract}{\normalsize\bfseries}
\newkomafont{authoraddendum}{\small}
%    \end{macrocode}
%
% \subsection{Abstract}
% The |abstract| environment is redefined to match the Word version.
%    \begin{macrocode}
\renewenvironment{abstract}{%
    \setparsizes{0pt}{4pt}{0pt plus 1 fill}{%
    \usekomafont{abstract}%
    \abstractname}\par%
    \usekomafont{abstractcontent}}{}
%    \end{macrocode}

% \subsection{Pagestyle}
% The demo document shows a headline containing author and document title. To achieve this that information was passed inside the |\maketitle| macro and saved in the |\MUC@author| and |\MUC@title| macros.
% There is also a possibility to set a copyright-information box similar to the Word template.
%    \begin{macrocode}
\ifMUC@pagestyle
	\PassOptionsToPackage{headsepline}{scrlayer-scrpage}
    \RequirePackage{scrlayer-scrpage}
    \ifMUC@copyrightinfo
%    \end{macrocode}
% Nowadays it is possible to write the complete content for the scrlayers directly inside the definition.
% To support older distributions as well the content is wrapped into a box.
% The most absolute length values have been extracted from different measurements to match the Word version.
%    \begin{macrocode}
        \newsavebox\MUC@copyrightbox
    	\begin{lrbox}{\MUC@copyrightbox}
    	\setlength{\fboxrule}{.7bp}%
    	\setlength{\fboxsep}{6.5pt}%
    	\fbox{\hspace*{.8pt}\parbox[t][\dimexpr2cm-2\fboxsep
            -2\fboxrule\relax][t]{%
                \dimexpr\textwidth-2\fboxsep-2\fboxrule%
                -1.6pt\relax}{%
    	Platzhalter für DOI und ggf. Copyright Text.
        (Bitte nicht entfernen).\par
    	\vspace{7pt}
    	Name, Vorname (2017): Titel.%
    	Tagungsband Mensch und Computer 2017.
        Gesellschaft für Informatik.%
    	DOI: xxxxxx}%
    	}
    	\end{lrbox}
    \DeclareNewLayer[align=tl,area={\dimexpr 1in
         + \hoffset + \oddsidemargin\relax}{%
        \dimexpr\paperheight-3.9cm\relax}{\textwidth}{2cm},
        contents={\usebox\MUC@copyrightbox}]{copyright}
    \DeclareNewPageStyleByLayers{MUC@titlepagestyle}{copyright}
    \else
    \DeclareNewPageStyleAlias{MUC@titlepagestyle}{empty}
    \fi
    \clearpairofpagestyles
    \ifMUC@headline
        \lohead{\MUC@title}
        \rehead{\csname MUC@author\endcsname}
    \fi
    \setkomafont{pagehead}{\normalfont}
    \setkomafont{pagefoot}{}
\fi
%    \end{macrocode}
%\iffalse
%</!font-clo>
%<*font-clo>
\ProvidesFile{mucfontsize10pt.clo}[2017/01/30 v1.0 mucproc font size class option (10pt)]

\def\normalsize{%
  \@setfontsize\normalsize{10}{12bp}
  \abovedisplayskip 10\p@ \@plus2\p@ \@minus5\p@
  \abovedisplayshortskip \z@ \@plus3\p@
  \belowdisplayshortskip 6\p@ \@plus3\p@ \@minus3\p@
  \belowdisplayskip \abovedisplayskip
  \let\@listi\@listI
}
\def\small{%
  \@setfontsize\small{9bp}{12bp}%
  \abovedisplayskip 8.5\p@ \@plus3\p@ \@minus4\p@
  \abovedisplayshortskip \z@ \@plus2\p@
  \belowdisplayshortskip 4\p@ \@plus2\p@ \@minus2\p@
  \def\@listi{\leftmargin\leftmargini
    \topsep 4\p@ \@plus2\p@ \@minus2\p@
    \parsep 2\p@ \@plus\p@ \@minus\p@
    \itemsep \parsep}%
  \belowdisplayskip \abovedisplayskip
}
\def\footnotesize{%
  \@setfontsize\footnotesize{\dimexpr18bp-10pt\relax}{\dimexpr21.6bp-12pt\relax}%
  \abovedisplayskip 6\p@ \@plus2\p@ \@minus4\p@
  \abovedisplayshortskip \z@ \@plus\p@
  \belowdisplayshortskip 3\p@ \@plus\p@ \@minus2\p@
  \def\@listi{\leftmargin\leftmargini
    \topsep 3\p@ \@plus\p@ \@minus\p@
    \parsep 2\p@ \@plus\p@ \@minus\p@
    \itemsep \parsep}%
  \belowdisplayskip \abovedisplayskip
}
\def\scriptsize{\@setfontsize\scriptsize{7}{8}}
\def\tiny{\@setfontsize\tiny{5}{6}}
\def\large{\@setfontsize\large{12bp}{14.4bp}}
\def\Large{\@setfontsize\Large{14bp}{18}}
\def\LARGE{\@setfontsize\LARGE{\dimexpr10pt + 6bp\relax}{\dimexpr12pt+7,2bp\relax}}
\def\huge{\@setfontsize\huge{20}{24}}
\def\Huge{\@setfontsize\Huge{24bp}{30}}
\normalsize
\setlength\footnotesep    {6.65\p@}
\setlength{\skip\footins} {9\p@ \@plus 4\p@ \@minus 2\p@}
\setlength\floatsep       {12\p@ \@plus 2\p@ \@minus 2\p@}
\setlength\textfloatsep   {20\p@ \@plus 2\p@ \@minus 4\p@}
\setlength\intextsep      {12\p@ \@plus 2\p@ \@minus 2\p@}
\setlength\dblfloatsep    {12\p@ \@plus 2\p@ \@minus 2\p@}
\setlength\dbltextfloatsep{20\p@ \@plus 2\p@ \@minus 4\p@}
\setlength\@fptop         {0\p@ \@plus 1fil}
\setlength\@fpsep         {8\p@ \@plus 2fil}
\setlength\@fpbot         {0\p@ \@plus 1fil}
\setlength\@dblfptop      {0\p@ \@plus 1fil}
\setlength\@dblfpsep      {8\p@ \@plus 2fil}
\setlength\@dblfpbot      {0\p@ \@plus 1fil}
\setlength\partopsep      {2\p@ \@plus 1\p@ \@minus 1\p@}
\def\@listi{\leftmargin\leftmargini
            \parsep 4\p@ \@plus2\p@ \@minus\p@
            \topsep 8\p@ \@plus2\p@ \@minus4\p@
            \itemsep4\p@ \@plus2\p@ \@minus\p@}
\let\@listI\@listi
\def\@listii {\leftmargin\leftmarginii
              \labelwidth\leftmarginii
              \advance\labelwidth-\labelsep
              \topsep    4\p@ \@plus2\p@ \@minus\p@
              \parsep    2\p@ \@plus\p@  \@minus\p@
              \itemsep   \parsep}
\def\@listiii{\leftmargin\leftmarginiii
              \labelwidth\leftmarginiii
              \advance\labelwidth-\labelsep
              \topsep    2\p@ \@plus\p@\@minus\p@
              \parsep    \z@
              \partopsep \p@ \@plus\z@ \@minus\p@
              \itemsep   \topsep}
\def\@listiv {\leftmargin\leftmarginiv
              \labelwidth\leftmarginiv
              \advance\labelwidth-\labelsep}
\def\@listv  {\leftmargin\leftmarginv
              \labelwidth\leftmarginv
              \advance\labelwidth-\labelsep}
\def\@listvi {\leftmargin\leftmarginvi
              \labelwidth\leftmarginvi
              \advance\labelwidth-\labelsep}
\endinput
%</font-clo>
%\fi
% \Finale
\endinput
